\pagestyle{plain}
\documentclass[12pt]{extreport}
\usepackage[a4paper, total={6in, 8in}]{geometry}
\usepackage{hyperref}
\usepackage{calrsfs}
\usepackage{fontspec,xunicode}
\usepackage{fontenc}
\usepackage{titlesec}
\usepackage{amsmath} 
\usepackage{amsfonts} 
\usepackage{graphicx}
\usepackage{float}
\usepackage{xeCJK}
\usepackage{listings}
\usepackage[framemethod=TikZ]{mdframed}
\usepackage{mathtools}
\usepackage{indentfirst}
\usepackage{amssymb}
\usepackage{ragged2e}
\DeclarePairedDelimiter\ceil{\lceil}{\rceil}
\DeclarePairedDelimiter\floor{\lfloor}{\rfloor}
\setCJKmainfont{AR PL UKai TW}
\lstset{basicstyle=\ttfamily,xleftmargin=24pt}
\titleformat{\section}[hang]{\normalfont\LARGE\bfseries}{\thesection}{0.5em}{}
\titleformat{\subsection}[hang]{\normalfont\large\bfseries}{ \thesubsection}{0.5em}{}
\setlength\parindent{24pt}
\setcounter{section}{-1}
\newcommand{\nequiv}{\not\equiv}
\newcommand{\Ih}{\hat{I}}
\DeclareMathAlphabet{\pazocal}{OMS}{zplm}{m}{n}
\newcommand{\Gc}{\pazocal{G}}
\setlength{\abovedisplayskip}{0pt}
\setlength{\belowdisplayskip}{0pt}
\setlength{\abovedisplayshortskip}{0pt}
\setlength{\belowdisplayshortskip}{0pt}
\linespread{1.2}
\setlength{\parskip}{0.5\baselineskip}
\begin{document}
\begin{center}
	\Huge{勉強算是講義的東西}\\ 
	\huge{Handsome Liu}\\ 
	\huge{2016.09.26}\\ 
\end{center}
\chapter{數與式}
    姑且解釋一下這章的標題:數跟式是兩個分開的東西。數是一個值,儘管它不一定是真實存在的,比方說我們會因為方便而想像出虛數,但它是一個具體而固定的值;而式裡面則可能出現一些不是數的文字,而這些文字可以用數代入。藉由把數代入文字,我們便賦予了式子一個特定的值。\\
    \section{數}
        如同前面說的,數是一個值。根據這些值的特性,有以下幾個重要的數的集合:自然數(\(\mathbb{N}\))、整數(\(\mathbb{Z}\))、有理數(\(\mathbb{Q}\))、無理數、實數(\(\mathbb{R}\))、複數(\(\mathbb{C}\))。\\
        \subsection{自然數}
            自然數就是正整數。至於正整數...就是正整數XD。難以簡潔的定義它。當然自然數在數學上是有嚴謹的定義的:\\
            \begin{mdframed}
            \(1\)是自然數。\\
            對於每一個自然數\(n\),\(n+1\)也是自然數。\\
            ...\\
            \end{mdframed}
            以上就是鼎鼎大名的皮亞諾公設,嚴謹的定義了自然數。不過反正不會考,所以當科普看過就好XD。當然我會提到這件事情是有原因的:它跟我們稍後會談到的數學歸納法息息相關。這兩個之間的關聯性等等再來聊。\\
            這一類的數之所以會自成一個集合,是因為他是人類在日常生活中首先需要面對的一種數。由於英文是 Natural Numbers 的關係,我們用符號\(\mathbb{N}\)來表示這個集合。\\
        \subsection{整數}
            整數就是自然數、負的自然數、跟\(0\)。儘管整數的英文是 Integers ,數學上是用\(\mathbb{Z}\)來表示。據說是因為德文的整數是 Zahlen ,天知道為什麼不用\(\mathbb{I}\),反正數學家都怪怪的。\\
        \subsection{有理數}
            有理數就有比較多東西可以拿來講了。首先,有理數是整數的比值。意思是說,一個數是\(q\)有理數,若且唯若有正整數\(m\)、\(n\)使得\(q=\dfrac{m}{n}\)。換句話說,有理數完完全全由整數建構出來,但整數是離散的,而有理數是稠密的。\\
            我們可以從很多方面來解讀有理數的稠密性。
\end{document} 
